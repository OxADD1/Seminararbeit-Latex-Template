% GRUNDLEGENDE PAKETE
% Zeichencodierung (UTF-8) für Umlaute und Sonderzeichen
\usepackage[utf8]{inputenc}
% Schriftcodierung für korrekte Darstellung
\usepackage[T1]{fontenc}
% Deutsche Sprachanpassung (neue Rechtschreibung)
\usepackage[ngerman]{babel}
% Grafikunterstützung für Bilder
\usepackage{graphicx}
% Seitenränder anpassen
\usepackage{geometry}
% Zeilenabstand einstellen
\usepackage{setspace}
% Kopf- und Fußzeilen anpassen
\usepackage{fancyhdr}
% Formatierung von Überschriften
\usepackage{titlesec}
% Anpassung des Inhaltsverzeichnisses
\usepackage{tocloft}

% WICHTIG: Xcolor VOR hyperref laden, sonst Konflikte
\usepackage{xcolor}
% Farbdefinition für Hyperlinks
\definecolor{hscolor}{RGB}{0,0,0}  % Schwarz für Links

% ZITIERUNGSKONFIGURATION (HARVARD-STIL)
% Unterstützung für Anführungszeichen gemäß deutscher Norm
\usepackage{csquotes}
% BibLaTeX mit Harvard-Stil (authoryear), natbib-Kompatibilität, Biber-Backend
\usepackage[style=authoryear,natbib=true,backend=biber]{biblatex}
% Datei mit Literatureinträgen einbinden
\addbibresource{literatur.bib}

% VERLINKUNG UND URLS
% Hyperref für PDF-Links und Metadaten (NACH xcolor laden)
\usepackage{hyperref}
% Verbessert PDF-Lesezeichen und Navigation
\usepackage{bookmark}
% Verbesserte URL-Darstellung
\usepackage{url}
% Zusätzliche Werkzeuge für LaTeX
\usepackage{etoolbox}

% SEITENLAYOUT
% Seitenränder definieren (25mm auf allen Seiten)
\geometry{
    a4paper,
    left=25mm,
    right=25mm,
    top=25mm,
    bottom=25mm
}

% Zeilenabstand 1,5 für bessere Lesbarkeit
\onehalfspacing

% ÜBERSCHRIFTENFORMATIERUNG
% Format für Abschnitt (section)
\titleformat{\section}{\normalfont\Large\bfseries}{\thesection}{1em}{}
% Format für Unterabschnitt (subsection)
\titleformat{\subsection}{\normalfont\large\bfseries}{\thesubsection}{1em}{}

% SEITENNUMMERIERUNG
% Plain-Stil für Seitenzahlen rechts unten (für Seiten mit Kapitelanfang)
\fancypagestyle{plain}{%
  \fancyhf{}%
  \fancyfoot[R]{\thepage}%
  \renewcommand{\headrulewidth}{0pt}%
}
% Standard-Seitenstil mit Seitenzahlen rechts unten
\pagestyle{fancy}
\fancyhf{}
\renewcommand{\headrulewidth}{0pt}
\fancyfoot[R]{\thepage}

% INHALTSVERZEICHNIS ANPASSUNGEN
% Schriftart für Überschrift des Inhaltsverzeichnisses
\renewcommand{\cfttoctitlefont}{\Large\bfseries}
% Ausrichtung nach dem Titel
\renewcommand{\cftaftertoctitle}{\hfill}

% Punktlinien für alle Verzeichnisebenen
\renewcommand{\cftsecleader}{\cftdotfill{\cftdotsep}}
\renewcommand{\cftsubsecleader}{\cftsecleader}

% Abstand nach Seitenzahl
\renewcommand{\cftsecafterpnum}{\cftparfillskip}
\renewcommand{\cftsubsecafterpnum}{\cftsecafterpnum}

% Einheitliche Einrückung für bessere Übersicht
\setlength{\cftsecnumwidth}{2em}
\setlength{\cftsubsecnumwidth}{2em}

% Keine Einrückung der Einträge
\setlength{\cftsecindent}{0pt}
\setlength{\cftsubsecindent}{0pt}

% HYPERREF KONFIGURATION
% Farben und Metadaten für PDF-Datei
\hypersetup{
    colorlinks=true,   % Farbige Links statt Rahmen
    linkcolor=hscolor, % Interne Links
    filecolor=hscolor, % Dateilinks
    urlcolor=hscolor,  % URL-Links
    citecolor=hscolor, % Zitierungen
    pdftitle={Seminararbeit},
    pdfauthor={},
    pdfsubject={},
    pdfkeywords={}
}

% VERZEICHNISKORREKTUR
% Korrigiert Abstand nach Nummerierung für einheitliches Layout
\makeatletter
\renewcommand{\numberline}[1]{#1\hspace{1em}}
\makeatother

% ABBILDUNGS- UND TABELLENVERZEICHNISKORREKTUR
% Seitenstil für Einträge im Abbildungsverzeichnis beibehalten
\makeatletter
\let\oldl@figure\l@figure
\def\l@figure#1#2{\oldl@figure{#1}{#2}\thispagestyle{fancy}}
\let\oldl@table\l@table
\def\l@table#1#2{\oldl@table{#1}{#2}\thispagestyle{fancy}}
\makeatother

% VERZEICHNISNAMEN ANPASSEN
% Deutsche Bezeichnungen mit römischen Nummerierungen
\renewcommand{\listfigurename}{II. Abbildungsverzeichnis}
\renewcommand{\listtablename}{III. Tabellenverzeichnis}

% Pfad für Bilder festlegen
\graphicspath{{images/}}