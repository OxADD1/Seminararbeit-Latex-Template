% HAUPTDATEI DER SEMINARARBEIT
\documentclass[12pt,a4paper,oneside]{article}

% Einbinden der Präambel mit allen Paketen und Konfigurationen
% GRUNDLEGENDE PAKETE
% Zeichencodierung (UTF-8) für Umlaute und Sonderzeichen
\usepackage[utf8]{inputenc}
% Schriftcodierung für korrekte Darstellung
\usepackage[T1]{fontenc}
% Deutsche Sprachanpassung (neue Rechtschreibung)
\usepackage[ngerman]{babel}
% Grafikunterstützung für Bilder
\usepackage{graphicx}
% Seitenränder anpassen
\usepackage{geometry}
% Zeilenabstand einstellen
\usepackage{setspace}
% Kopf- und Fußzeilen anpassen
\usepackage{fancyhdr}
% Formatierung von Überschriften
\usepackage{titlesec}
% Anpassung des Inhaltsverzeichnisses
\usepackage{tocloft}

% WICHTIG: Xcolor VOR hyperref laden, sonst Konflikte
\usepackage{xcolor}
% Farbdefinition für Hyperlinks
\definecolor{hscolor}{RGB}{0,0,0}  % Schwarz für Links

% ZITIERUNGSKONFIGURATION (HARVARD-STIL)
% Unterstützung für Anführungszeichen gemäß deutscher Norm
\usepackage{csquotes}
% BibLaTeX mit Harvard-Stil (authoryear), natbib-Kompatibilität, Biber-Backend
\usepackage[style=authoryear,natbib=true,backend=biber]{biblatex}
% Datei mit Literatureinträgen einbinden
\addbibresource{literatur.bib}

% VERLINKUNG UND URLS
% Hyperref für PDF-Links und Metadaten (NACH xcolor laden)
\usepackage{hyperref}
% Verbessert PDF-Lesezeichen und Navigation
\usepackage{bookmark}
% Verbesserte URL-Darstellung
\usepackage{url}
% Zusätzliche Werkzeuge für LaTeX
\usepackage{etoolbox}

% SEITENLAYOUT
% Seitenränder definieren (25mm auf allen Seiten)
\geometry{
    a4paper,
    left=25mm,
    right=25mm,
    top=25mm,
    bottom=25mm
}

% Zeilenabstand 1,5 für bessere Lesbarkeit
\onehalfspacing

% ÜBERSCHRIFTENFORMATIERUNG
% Format für Abschnitt (section)
\titleformat{\section}{\normalfont\Large\bfseries}{\thesection}{1em}{}
% Format für Unterabschnitt (subsection)
\titleformat{\subsection}{\normalfont\large\bfseries}{\thesubsection}{1em}{}

% SEITENNUMMERIERUNG
% Plain-Stil für Seitenzahlen rechts unten (für Seiten mit Kapitelanfang)
\fancypagestyle{plain}{%
  \fancyhf{}%
  \fancyfoot[R]{\thepage}%
  \renewcommand{\headrulewidth}{0pt}%
}
% Standard-Seitenstil mit Seitenzahlen rechts unten
\pagestyle{fancy}
\fancyhf{}
\renewcommand{\headrulewidth}{0pt}
\fancyfoot[R]{\thepage}

% INHALTSVERZEICHNIS ANPASSUNGEN
% Schriftart für Überschrift des Inhaltsverzeichnisses
\renewcommand{\cfttoctitlefont}{\Large\bfseries}
% Ausrichtung nach dem Titel
\renewcommand{\cftaftertoctitle}{\hfill}

% Punktlinien für alle Verzeichnisebenen
\renewcommand{\cftsecleader}{\cftdotfill{\cftdotsep}}
\renewcommand{\cftsubsecleader}{\cftsecleader}

% Abstand nach Seitenzahl
\renewcommand{\cftsecafterpnum}{\cftparfillskip}
\renewcommand{\cftsubsecafterpnum}{\cftsecafterpnum}

% Einheitliche Einrückung für bessere Übersicht
\setlength{\cftsecnumwidth}{2em}
\setlength{\cftsubsecnumwidth}{2em}

% Keine Einrückung der Einträge
\setlength{\cftsecindent}{0pt}
\setlength{\cftsubsecindent}{0pt}

% HYPERREF KONFIGURATION
% Farben und Metadaten für PDF-Datei
\hypersetup{
    colorlinks=true,   % Farbige Links statt Rahmen
    linkcolor=hscolor, % Interne Links
    filecolor=hscolor, % Dateilinks
    urlcolor=hscolor,  % URL-Links
    citecolor=hscolor, % Zitierungen
    pdftitle={Seminararbeit},
    pdfauthor={},
    pdfsubject={},
    pdfkeywords={}
}

% VERZEICHNISKORREKTUR
% Korrigiert Abstand nach Nummerierung für einheitliches Layout
\makeatletter
\renewcommand{\numberline}[1]{#1\hspace{1em}}
\makeatother

% ABBILDUNGS- UND TABELLENVERZEICHNISKORREKTUR
% Seitenstil für Einträge im Abbildungsverzeichnis beibehalten
\makeatletter
\let\oldl@figure\l@figure
\def\l@figure#1#2{\oldl@figure{#1}{#2}\thispagestyle{fancy}}
\let\oldl@table\l@table
\def\l@table#1#2{\oldl@table{#1}{#2}\thispagestyle{fancy}}
\makeatother

% VERZEICHNISNAMEN ANPASSEN
% Deutsche Bezeichnungen mit römischen Nummerierungen
\renewcommand{\listfigurename}{II. Abbildungsverzeichnis}
\renewcommand{\listtablename}{III. Tabellenverzeichnis}

% Pfad für Bilder festlegen
\graphicspath{{images/}}

% BEGINN DES DOKUMENTS
\begin{document}

% TITELSEITE (ohne Seitennummer)
\pagenumbering{gobble} % Keine Seitenzahlen
\begin{titlepage}
    \begin{center}
        % Hochschullogo
        \includegraphics[width=10cm]{hs-albsig-logo.png}\\[3cm]

        % Titel in Rahmen
        \begin{tabular}{c}
            \framebox{
                \begin{minipage}{0.8\textwidth}
                    \vspace{1cm}
                    \begin{center}
                        \LARGE\textbf{DAS IST DIE SEMINARARBEIT}

                        \vspace{0.5cm}

                        \Large{Seminararbeit}
                    \end{center}
                    \vspace{1cm}
                \end{minipage}
            }
        \end{tabular}\\[2cm]

        % Verfasserinformationen mit automatischem Datum durch \today
        \begin{tabular}{rl}
            eingereicht am: & \today                     \\
            von:            & Vorname Nachname           \\
                            & geboren am XX. Monat XXXX  \\[0.5cm]
            Matrikelnummer  & XXXXXXX                    \\[0.5cm]
            Betreuer        & Prof. Dr. Vorname Nachname \\
        \end{tabular}

        \vfill

        % Trennlinie
        \rule{\textwidth}{0.5pt}

        \vspace{0.5cm}

        % Hochschulinformationen
        \begin{tabular}{c}
            Hochschule Albstadt-Sigmaringen          \\
            Fakultät Informatik                      \\
            Telefon: +49 7571-7320                   \\
            Internet: \url{https://www.hs-albsig.de} \\
        \end{tabular}
    \end{center}
\end{titlepage}
\clearpage

% INHALTSVERZEICHNIS (ohne Seitenzahl)
% Wir behalten gobble für keine Seitenzahlen bei
\renewcommand{\contentsname}{Inhaltsverzeichnis}
\tableofcontents
\clearpage

% ABKÜRZUNGSVERZEICHNIS (beginnt mit römisch i)
\pagenumbering{roman} % Setzt römische Ziffern (i, ii, iii, ...)
\setcounter{page}{1} % Beginnt mit i
\pagestyle{fancy}
\phantomsection
\section*{I. Abkürzungsverzeichnis}
\addcontentsline{toc}{section}{I. Abkürzungsverzeichnis}
% Hier Abkürzungen eintragen
% Beispielstruktur für Abkürzungen:

\begin{tabular}{ll}
    %Abk. & Bedeutung \\
    %z.B. & zum Beispiel \\
    %usw. & und so weiter \\
\end{tabular}
\clearpage

% ABBILDUNGSVERZEICHNIS (sollte automatisch Seite ii sein)
\phantomsection
\addcontentsline{toc}{section}{II. Abbildungsverzeichnis}
\listoffigures
\thispagestyle{fancy}
\clearpage

% TABELLENVERZEICHNIS (sollte automatisch Seite iii sein)
\phantomsection
\addcontentsline{toc}{section}{III. Tabellenverzeichnis}
\listoftables
\thispagestyle{fancy}
\clearpage

% HAUPTTEIL BEGINNT - Mit arabischen Zahlen ab Seite 1
\pagenumbering{arabic}
\setcounter{page}{1}
\pagestyle{fancy}

% Einführung
\phantomsection
\section{Einführung}

\subsection{Aufgabenstellung}
% Hier Text zur Aufgabenstellung einfügen
% Harvard-Zitation in Klammern: \parencite{mueller2020}
% Harvard-Zitation im Text: Laut \textcite{schmidt2022} ist...

\subsection{Forschungsfragen}
% Hier Forschungsfragen formulieren

\subsection{Aufbau der Arbeit}
% Hier Struktur der Arbeit beschreiben
\clearpage

% Kapitel 2 
\phantomsection
\section{Kapitel 2}
% Hier Inhalt für Kapitel 2 einfügen

\subsection{Unterabschnitt 2.1}
% Hier Text für Unterabschnitt 2.1 einfügen

\subsection{Unterabschnitt 2.2}
% Hier Text für Unterabschnitt 2.2 einfügen
\clearpage

% Kapitel 3 
\phantomsection
\section{Kapitel 3}
% Hier Inhalt für Kapitel 3 einfügen

\subsection{Unterabschnitt 3.1}
% Hier Text für Unterabschnitt 3.1 einfügen

\subsection{Unterabschnitt 3.2}
% Hier Text für Unterabschnitt 3.2 einfügen
\clearpage


% Fazit
\phantomsection
\section{Fazit}

\subsection{Zusammenfassung}
% Hier Zusammenfassung der Ergebnisse einfügen

\subsection{Beantwortung der Forschungsfragen}
% Hier auf Forschungsfragen eingehen

\subsection{Ausblick}
% Hier Ausblick auf zukünftige Entwicklungen geben
\clearpage

% LITERATURVERZEICHNIS - Mit römischen Zahlen
\phantomsection
\pagenumbering{Roman} % Großgeschriebene römische Zahlen
\setcounter{page}{1}
\pagestyle{fancy}
\section*{Literaturverzeichnis}
\addcontentsline{toc}{section}{Literaturverzeichnis}
\printbibliography[heading=none]

% EIGENSTÄNDIGKEITSERKLÄRUNG
\phantomsection
\clearpage
% Römische Nummerierung läuft automatisch weiter
% Erstellt eine Überschrift "Eigenständigkeitserklärung" ohne Nummerierung durch den *
\section*{Eigenständigkeitserklärung}
% Fügt die Eigenständigkeitserklärung ins Inhaltsverzeichnis ein, obwohl sie keine Nummer hat
\addcontentsline{toc}{section}{Eigenständigkeitserklärung}

% Fügt 1 cm vertikalen Abstand ein, um den Text von der Überschrift zu trennen
\vspace{1cm}

% Der eigentliche Text der Eigenständigkeitserklärung
Hiermit erkläre ich, dass ich die vorliegende Arbeit selbstständig und ohne fremde Hilfe verfasst und keine anderen als die angegebenen Hilfsmittel verwendet habe. Alle Stellen oder Passagen der vorliegenden Arbeit, die anderen Quellen entnommen wurden, sind als solche kenntlich gemacht. Diese Arbeit wurde in gleicher oder ähnlicher Form noch keiner Prüfungsbehörde vorgelegt.

% Fügt 2 cm vertikalen Abstand ein für den Unterschriftsbereich
\vspace{2cm}

% Verhindert Einrückung des folgenden Textes
\noindent
% Erstellt eine Tabelle mit zwei linksbündigen Spalten (ll)
\begin{tabular}{ll}
% Erzeugt eine horizontale Linie nur unter der ersten Spalte
\cline{1-1}
% Erster Eintrag "Ort, Datum", dann Spaltentrenner &, dann Zeilenumbruch mit 1cm Abstand nach unten
Ort, Datum & \\[1cm]
% Zweite horizontale Linie unter der ersten Spalte
\cline{1-1}
% Zweiter Eintrag "Unterschrift" und Zeilenende
Unterschrift & \\
\end{tabular}

% ======== UM EIN BILD MIT UNTERSCHRIFT EINZUFÜGEN ========
% Variante 1: Unterschrift als eingescanntes Bild
% Füge folgenden Code anstelle der Tabelle ein:
% \noindent
% \begin{tabular}{ll}
% \cline{1-1}
% Ort, Datum & \\[1cm]
% \cline{1-1}
% \raisebox{-0.5cm}{\includegraphics[width=4cm,height=1cm]{images/unterschrift.png}} & \\
% \end{tabular}
%
% Hierbei musst du:
% 1. Deine Unterschrift einscannen und als PNG/JPG speichern
% 2. Die Datei in den images-Ordner legen
% 3. Die Größe (width=4cm,height=1cm) an deine Unterschrift anpassen
% 4. Mit \raisebox kannst du die vertikale Position anpassen
%
% Variante 2: Platz für handschriftliche Unterschrift lassen
% Die aktuelle Lösung ist bereits gut dafür geeignet, 
% da sie Platz für eine handschriftliche Unterschrift bietet.
% Du druckst das Dokument aus und unterschreibst dann von Hand.

\end{document}